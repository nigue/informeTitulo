%definición del artículo
\documentclass[a4paper,12pt,openany,oneside]{book}
\usepackage[left=5cm,right=2cm,top=4cm,bottom=4cm,paperwidth=216mm,paperheight=330mm,pdftex]{geometry}
%paquete usado para silabación en español
\usepackage[spanish]{babel}
%codificación del documento
\usepackage[utf8]{inputenc}
%espaciado
\linespread{1.5}
%identación de párrafo
\setlength{\parindent}{20pt}
%espaciado de párrafo
\setlength{\parskip}{4ex plus 0.5ex minus 0.2ex}
%para validar sólo sintaxis sin compilar
%\usepackage{syntonly}
%\syntaxonly
%para usar imágenes
\usepackage{graphicx}
%para usar fragmentos de codigo fuente
\usepackage{listings}
\usepackage{float}
\floatstyle{boxed}
\newfloat{codigo}{thp}{lop}
\floatname{codigo}{Caja de Código}
%comienzo del documento
\begin{document}
\thispagestyle{empty}
\begin{center}
\textbf{UNIVERSIDAD TECNOLÓGICA METROPOLITANA\\
ESCUELA DE INFORMÁTICA}\\
\vspace{3cm}
SISTEMA ESTADÍSTICO DE PROFESORES Y ALUMNOS\\S.E.P.A.
\end{center}
\begin{flushright}
TRABAJO DE TÍTULO PARA OPTAR AL\\
TÍTULO DE INGENIERO CIVIL EN\\
COMPUTACIÓN\\
MENCIÓN INFORMÁTICA.\\
\vspace{3cm}
PROFESOR GUÍA: Sebastián Salazar Molina\\
\vspace{1.5cm}
Miguel Ángel Aníbal Davor Fuenzalida Pino
\end{flushright}
\vspace{4cm}
\begin{center}
SANTIAGO - 2013
\end{center}
\newpage
\thispagestyle{empty}
\begin{flushright}
\vspace{20mm}
Nota: \line(1, 0){140} \\
\vspace{30 mm}
\line(1, 0){180}\\	
Firma y Timbre\\
Autoridad Responsable
\end{flushright}
\chapter*{Resumen}
\thispagestyle{empty}
Este trabajo consiste en la estructuración del proyecto SEPA, enfocado al control de información, perteneciente a la Universidad Tecnológica Metropolitana. A través de tecnologías potentes que nos permiten un diseño moderno y flexible, a su vez nos facilitan el proceso de creación de nuevas entidades y características. Permitiendo alargar la vida de este proyecto. Cabe destacar que el proyecto se hará bajo la metodología PSP que nos proporciona un modelo de trabajo basado en la caracterización de los tiempos y costes del trabajo de un ingeniero informático. De esta manera, este proyecto es una nueva forma de potenciar las capacidades operativas de la Universidad, manejando de mejor manera su calidad y competitividad dentro del mercado universitario.

Para hacer este proyecto se establece un modelo de trabajo dado por la metodología PSP (Personal Software Process) y adicionando documentación extra que está en el apartado Anexos, la cual responde a eventos que caracterizaron las directrices de cómo se lleva adelante este trabajo. También es importante destacar que, como es un trabajo de renovación, este proyecto tendrá una nueva forma y permitirá a más usuarios poder acceder a información de la universidad.

Finalmente este proyecto está enfocado en ser un ejemplo para potenciar las capacidades administrativas de la Universidad, haciendo un ente más reacio a mejorar. Por la calidad de la educación y la excelencia en esta tarea.
\tableofcontents
\listoffigures
\chapter*{Introducción}
\thispagestyle{empty}
La información es un recurso sumamente importante en las organizaciones. Ésta permite hacer la toma de decisiones y la manipulación de los recursos físicos. Sin embargo, teniendo la información por si sola, no es posible hacer un buen análisis a futuro, por lo que los usuarios de estos datos necesitan datos procesados de manera mas comprensible, información que le permita responder rápidamente el qué hacer de ahora en adelante, todo esto bajo el contexto universitario. La cantidad de información va creciendo a medida que la universidad tiene mas usuarios dentro de su organización, ya sean alumnos, profesores o encargados, también aumenta la complejidad de sostener la información. Lo que sugiere dos cosas, entre más información es más probable que los valores finales de los datos procesados sean mayormente fidedignos, con mayores volúmenes de información se espera tener sistemas de manejo de mayor envergadura y capacidad. Razón por la cual la información se ha vuelto un recurso tan necesario e importante en todo tipo de organización, ya sea una empresa publica como una universidad o una empresa privada.

La Universidad Tecnológica Metropolitana tiene varios sistemas informativos que permiten gestionar los sistemas docentes de su organización, pero sin embargo faltan proyectos que midan la calidad del trabajo docente que se imparte, ya sea desde la perspectiva de un alumno, funcionario y finalmente la perspectiva de un docente frente a su propio trabajo. Para estos efectos se requieren enormes cantidades de datos institucionales que la universidad posee, pero no dispone de un sistema interno que permita la correcta toma de decisiones mediante el uso de los indicadores que toda la actual información puede entregar. El sistema que más indicadores usa es el SEPA, el cual permite a la universidad establecer directrices con respecto a varios indicadores que arroja el sistema. Este proyecto fue realizado por Sebastián Salazar Molina para su Trabajo de Título y corresponde a lo que la universidad tiene para medir su calidad a través de información procesada. Actualmente el Sistema no tiene conexión con la información proveniente de otro sistema llamado DirDoc, el cual contiene las actas de notas para los estudiantes de la Universidad. Esta característica que aún no existe, está dentro de un compendio de necesidades que deben ser satisfechas en pos de ayudar a la situación actual de la Universidad en estos periodos donde necesita mostrar su calidad y su competitividad frente a otros organismos universitarios.

En este proyecto se propone realizar una renovación total del sistema SEPA incluyendo mejoras que puedan salir de la toma de requerimientos, siendo parte de un conjunto de cosas que deben existir como mínimo, agregando también requerimientos que puedan ser útiles para hoy en día. También incluir la capacidad de que pueda ser mejorado en los siguientes años, sin mayor complejidad.

Para cumplir este propósito se propone la estructuración del código con java, lo que nos permite hacer uso de herramientas que nos facilitarán el desarrollo y el diseño del proyecto. Es importante destacar que ocuparemos tecnologías que básicamente son recientes, y que nos permiten diseñar aplicaciones con el estándar Modelo Vista Controlador de manera rápida y ágil, para este propósito existen herramientas que nos facilitarán la comunicación entre las distintas capas del proyecto, facilitando aún más la puesta en marcha de este proyecto.

Para el desarrollo íntegro de este proyecto se hará uso de una metodología de desarrollo de proyectos de manera individual, perfecto para trabajos de Título como el que se está realizando. La metodología PSP (Personal Software Process) establece aristas de desempeño que permiten a su usuario responder a patrones de trabajo que en un futuro serán la base para calcular costes y tiempos que un ingeniero necesita para la realización de sus actividades de manera rápida y vital.

Existen 3 tipos de beneficios que una organización puede obtener al hacerse con la versión mejorada del sistema SEPA. El primero corresponde al beneficio interno que se produce, este conlleva una importante evolución de como se ve la organización por dentro, ya que se mostrará a la universidad como un ente preocupado de medir de manera cuantificable su calidad como organización. El segundo corresponde al beneficio de los usuarios de estos sistemas modernos, los cuales se ven favorecidos, ya que tendrán a su disposición una completa herramienta que los apoyará en sus actividades, de la mejor manera posible. Finalmente el beneficio a futuro, para quienes se den la tarea de complementar este sistema o mejorarlo, ya que no les va a ser más difícil la implementación de mejoras y cambios.

Para este proyecto en particular se espera una pronta puesta en marcha, ya que es bastante necesario para la Universidad como para los usuarios finales de este sistema. Esto apoya el concepto de estar con una mirada innovadora frente a proyectos que potencian las capacidades de generar prontas respuestas dentro de la organización. Para los usuarios, el sistema va a ser un complemento de enorme ayuda, de manera rápida, segura y fácil.

En el Capítulo 1 se tratan los antecedentes generales que hacen del proyecto una necesidad hacia la universidad y cómo estas necesidades se transforman en problemas que deben tener una solución concreta y rápida, con el fin de cumplir ciertos logros académicos dentro de la organización.

En el Capítulo 2 se puede ver las características principales de la organización y cómo se encuentra a la hora de recibir un proyecto que potenciará su capacidad de evaluación de excelencia y calidad.

En el Capítulo 3 se muestran las características principales del problema y cómo éste debe ser solucionado, de qué manera nosotros como estudiantes podemos dar una solución lógica que imprima calidad a los procesos productivos de la universidad.

En el Capítulo 4 se explaya en los límites de la solución, la cual está en el marco de un proyecto de trabajo de título, lo que quiere decir que hay ciertas restricciones y alcances a la hora de abordar este tipo de proyectos.

En el Capitulo 5 se relata un poco de historia de las herramientas que permiten la pronta realización de las actividades de desarrollo.
\chapter{Antecedentes}
\thispagestyle{empty}
\section{Contexto General}
El proyecto que este informe de Título expone esta inmerso sobre la Universidad Tecnológica Metropolitana (UTEM), la cual es un organismo de enseñanza superior que se encuentra dentro del Consejo de Rectores de las Universidades Chilenas. Su característica principal es ser una de las primeras en inmersión de la primera generación de estudiantes universitarios para las familias de clase media, en comparación a otras universidades estatales.

La importancia de esta universidad radica en que es una universidad que tiene el mayor porcentaje de alumnos pertenecientes a la primera generación universitaria dentro de sus familias. Por lo cual el compromiso con la calidad es fundamental, mostrando un buen nivel educativo frente a otras universidades estatales. Finalmente a la UTEM se le adjudica, por sobre todo, una gran preparación de profesionales en el área tecnológica de las TIC (Tecnologías de Información y Comunicación).

Esto requiere medidas de certificación de calidad, para la organización en general, por lo que se necesita claramente la acreditación de la universidad y sus carreras.
\section{Contexto Particular}
Actualmente la universidad se encuentra en un nuevo proceso de acreditación, lo que responde a una necesidad de poseer herramientas que permitan la evaluación de la calidad docente y administrativa de la universidad.

El proyecto SEPA, Sistema Estadístico de Profesores y Alumnos, es el encargado de mostrar información que pueda ser cuantificable en pos de tener herramientas de control general y toma de decisiones.
\section{Problema}
Para la nueva acreditación (de la comisión acreditadora) que se encuentra en preparación, ya a comienzos del año 2013, se necesita la reingeniería del proyecto SEPA, el cual sera una versión renovada que mostrará mas características y permitirá una mejor forma de evaluar la calidad docente y administrativa de la universidad.

El problema en este caso es la realización de estas actividades a favor de los procesos administrativos que requieren los organismos que evalúan y acrediten a la Universidad Tecnológica Metropolitana. De esta manera se pone en discusión la pronta necesidad de tener una herramienta que represente una versión mejorada de el sistema anterior SEPA.
\section{Motivación}
Uno de los primeros motivos que aparece, es el hecho de ayudar a la Universidad, con un proyecto que potencie las capacidades de ésta. Convirtiéndola en un organismo más competitivo en el mercado e impulsando el desarrollo de sus capacidades. Esperando que con el tiempo se dé la posibilidad de crear nuevas cosas a partir del paso que este proyecto puede generar, esto quiere decir que la Universidad debe estar más abierta a recibir proyectos internos, que sean hechos por sus propios alumnos, quienes son parte fundamental del desarrollo estudiantil y conocen de mejor manera las cualidades y características de su Universidad. Este proyecto tiene como motivación generar una herramienta de apoyo que sea de calidad y que beneficie los procesos productivos.

La motivación de parte del alumno comprende el aprendizaje de tecnologías, metodologías y herramientas que le permiten en un futuro tener mejores oportunidades y ser más competitivo en el ámbito laboral. Pues esto supone la composición de un sistema, el cual requiere dominar ciertas tecnologías de manera acabada. Esta motivación es importante ya que no sólo es una experiencia de aprendizaje, si no que también supone una manera de ver desafíos propios, tales como proyectos propios que el alumno pueda tener en algún futuro. Pensando de esta manera es lógico suponer el hecho de que el alumno tendrá la necesidad de realizar proyectos en pos de mejorar las condiciones de organizaciones benefactoras para el estado y también instituciones de educación universitaria.
\chapter{La Organización}
\thispagestyle{empty}
\section{Misión}
La Universidad Tecnológica Metropolitana es una organización estatal docente enfocada en la generación de profesionales que estén a la altura de los proyectos que se necesitan en pos de mejorar el país de manera sólida, segura y tecnológica. Para este motivo transforma a personas en profesionales con altas capacidades para enfrentar problemas. Estos profesionales se caracterizan por la fuerte convicción de lograr sus objetivos incluso en situaciones difíciles.
\section{Visión}
La organización será un ente de enseñanza que contará con una inclusión de profesionales de un alto nivel académico, mejorando así su capacidad de formar los profesionales del futuro. Tendrá áreas de investigación y creación tecnológica que permitirán la modernización en muchas áreas del conocimiento. Dentro de la organización existirá una amplia aceptación de inclusión de a cualquier tipo de personas ya sea por su raza, etnia o estrato social. 
\section{Organigrama Empresarial}

\section{Estructura de desarrollo el proyecto}
El desarrollo del proyecto esta bajo la metodología de desarrollo Personal Sofware Process (PSP), la cual permite medir en forma numérica todas la variables implicadas en el proceso productivo de la creación de este proyecto, lo que supone una estructuración lógica de las actividades que se realizan en pos de este trabajo, para lograr este propósito se debe hacer un esfuerzo extra, a su vez será recompensado con valiosa información tanto del proyecto como del alumno o trabajador involucrado en el proceso de desarrollo.
\chapter{Problema}
\thispagestyle{empty}
\section{Situación Actual}
La organización dispone del proyecto anterior llamado SEPA, para estos efectos dispone de una paleta de funcionalidades, que esperan en un futuro poder mejorar. Mientras tanto se pueden generar gráficos y variables estadísticas que muestran información útil a la hora de tomar decisiones. Para estos efectos ya se han pedido mejoras e inclusión de más actores frente a este sistema. Agregando también el hecho de que pronto la universidad se enfrenta a un nuevo proceso de acreditación lo que supone una mejora de este sistema en particular.
\section{Situación Propuesta}
La universidad dispone de un sistema de análisis estadístico que permite la toma de decisiones de manera más especializada, enfocada a los distintos tipos de usuario que pueden existir en el conjunto de la organización. Existe más inclusión ya que participan más personas y más información. El sistema  es parte de las características que la organización tiene a la hora de enfrentar los desafíos administrativos y docentes. Incluyendo también la participación exitosa de la universidad en los procesos de acreditación.
\section{Objetivos Generales y Específicos}
\subsection{Objetivos Generales}
\textit{Desarrollar un sistema de medición estadística para la Universidad Tecnológica Metropolitana ayudando a sus procesos docentes y productivos, mejorando la calidad de la educación y los niveles de enseñanza académica que la organización desea potenciar.}
\begin{enumerate}
\item Un objetivo general es la creación de un sistema web basado en el estándar Modelo Vista Controlador, asegurando un producto que sea eficiente en el cumplimiento de sus objetivos. Para esto se espera que el sistema tenga a lo menos un grado de sustentabilidad que lo haga eficiente, rápido y cómodo a la hora de usar, para todos los tipos de usuarios que se establecerán. Las características principales serán que este sistema estará hecho en Java lo que nos permite una fácil implementación de tecnologías, las cuales nos ayudarán en el proceso de reconstrucción de un proyecto en el marco de las TIC, especialmente el de la creación de proyectos web.
\item El otro objetivo principal es la utilización de una metodología de desarrollo, en este caso PSP (Personal Software Process), la cual nos permite llevar un seguimiento de las actividades que se deben realizar, de manera lógica y ordenada, para este propósito la metodología propone la utilización de valores contables en el tiempo, estos valores establecen las características y condiciones del trabajo del ingeniero. Por lo cual, se tendrá en cuenta todo un estudio con respecto al trabajo que se realiza y será parte integral del proyecto de Trabajo de Título.
\end{enumerate}
\subsection{Objetivos Específicos}
\begin{enumerate}
	\item Desarrollar un sistema que sea eficiente para todos los tipos de usuario.
	\item Facilitar la creación o la producción de mejoras sobre este nuevo sistema.
	\item Mostrar el desarrollo del proyecto a través de una metodología de desarrollo como lo es PSP.
	\item Potenciar las capacidades de la universidad, entregando un producto de calidad para una organización enfocada en la producción de profesionales del futuro.
\end{enumerate}
\section{Esquema de trabajo}
Para la realización de las actividades pertinentes al desarrollo del proyecto y la redacción de documentos asociados al trabajo de título, se tiene un modelo de trabajo aceptado por el alumno y el profesor encargado del ramo. Este modelo dicta que debe existir una reunión semanal para testificar los avances y las nuevas peticiones que se van agregando al proyecto a medida avanza. Para este propósito se establece un día, un lugar y una hora de la semana. Esto resulta positivo para el proyecto ya que todas las semanas hay oportunidad de reparar alguna arista del proyecto que esté mal encaminada, proporcionando información útil en tiempos cortos, como lo es una revisión sobre un documento o un progreso del proyecto.

Cabe señalar que el trabajo que realiza el alumno también esta medido a través de un esquema de trabajo, éste esta dado por la metodología de desarrollo PSP, en la cual el estudiante va anotando sus tiempos de respuesta a las distintas actividades que tenga que realizar, ya sea en el trabajo de título como en el proyecto. Para en un futuro tener información de utilidad que muestre de manera justificada la calidad del trabajo.
\chapter{Alcances y Limitaciones}
\thispagestyle{empty}
\section{Alcances del Proyecto}
El primer compromiso, de carácter general, que se establece para este proyecto, es la estructuración del proyecto SEPA en el lenguaje Java. Esto indica la necesidad de ocupar tecnologías que rodean al convención de diseño llamada J2EE, la cual nos proporciona herramientas que facilitan el desarrollo de aplicaciones MVC (Modelo Vista Controlador) a través de sus componentes. A esto hay que agregar el hecho de que se tendrán en cuenta otras tecnologías que nos facilitan el manejo de los datos, como lo es Spring, que resulta importante en la creación de proyectos Web basados en Java.

Como segundo alcance, el proyecto estará enfocado en la coparticipación de indicadores UTIGRA (Unidad de Títulos y Grados), esto quiere decir que de aquí en adelante también trabajaremos con información pertinente a gente que ha terminado sus procesos estudiantiles, mostrándonos un nuevo abanico de información. De esta manera terminamos con una arista importantísima a la hora de establecer un marco de trabajo.

Para el proyecto en particular se tendrán en cuenta la creación de una nueva cantidad de perfiles, adaptándose a las nuevas necesidades y sobre todo al cambio con relación a UTIGRA. Para esto se incorporan más perfiles de los que se tenían en un principio, aumentando la cantidad de usuarios del sistema y ampliando sus posibilidades.
\section{Limitaciones del Proyecto}
Por otro lado, cabe mencionar que el proyecto no contendrá requerimientos que se hagan fuera de plazo, esto es importante ya que como el trabajo que se realiza es parte del desarrollo que se ve sustentado a través de una metodología de desarrollo como lo es PSP, no es conveniente agregar más requerimientos. Esto no quiere decir que la metodología no acepte la inclusión de requerimientos a mitad de trabajo, pero es necesario establecer esa política en el marco de un trabajo de Título.
\section{Alcances del Trabajo de Título}
Los alcances del Trabajo de Título están delimitados por las características que el trabajo de título debe tener, es decir, que se hace un marco de trabajo que comprende los tópicos a trabajar. En enmarcar esta cantidad de trabajo acotado nos permite tener una medida para poder evaluar con una calificación al estudiante. También nos permite el seguimiento de las actividades que el estudiante realiza a medida avanza en su proyecto. Por tanto, los alcances, serán los requerimientos que sean aprobados por el profesor que dicta el ramo.
\section{Limitaciones del Trabajo de Título}
El trabajo no comprende áreas que estén fuera de los límites temporales a la toma de requerimientos, por lo cual es necesario indicar que hay características y funcionalidades que a pesar de ser levantadas puede que no sean parte de este trabajo de título. Por tanto es necesario destacar que hay un número acotado de entidades y actividades que respaldan con argumentos la elección que se tomo entre el profesor y el alumno.
\chapter{Marco Teórico}
\thispagestyle{empty}
\section{Base Conceptual}
\subsection{Establecer el problema}
En la universidad existe la necesidad de mostrar información relevante para la toma de decisiones, esta información debe ser respaldada por las grandes cantidades de registros que se guardan constantemente en la universidad. Para este propósito existe un sistema llamado Sistema Estadístico de Profesores y Alumnos (SEPA), el cual otorga información importante para la toma de decisiones y verificar la calidad de la educación. Pero como la universidad ha crecido, también crece la cantidad de información y la necesidad de tener datos más fuertes y de manera más rápida. Desde este punto se plantea la necesidad de renovar el sistema SEPA, para cumplir con las exigencias que hoy en día se necesitan completar. Entre las más necesarias se encuentran: La creación de distintos tipos de usuarios, tal que cada uno tenga una paleta de actividades distintas según su cargo; Mostrar el sistema como parte del proceso de acreditación de la universidad.
\subsection{Establecer la Solución}
Se opta por una completa reingeniería del proyecto, en este caso se utilizará Java para generar el código de este sistema. Se mantendrán algunas cosas que vienen ya hechas en el proyecto anterior, pero no se asegura la integridad total de las estructuras anteriores, esto quiere decir que si es necesario se hará un cambio fuerte en la base de datos para poder sobrellevar los nuevos requerimientos, estos requerimientos y los anteriores estarán en armonía.
\subsection{¿Por qué ocupar Java?}
El proyecto es construido a través del compendio de tecnologías que engloba la convención J2EE, ya que de esta manera nos enfocamos en el proceso productivo de una manera bastante rígida, gracias a esta forma de trabajo podemos automatizar varios procesos y trabajar de una manera bastante veloz en un sin fin de cosas que nos facilitan el trabajo y el desarrollo del proyecto. En la actualidad no sólo J2EE, si no también otras tecnologías asociadas a Java, eso facilita la reconstrucción del proyecto de manera más expedita y rápida. Todo con el fin de realizar de la mejor manera posible este proyecto que beneficia a la universidad.
\subsection{¿Por qué no otras tecnologías?}
En comparación con PHP el cual nos permite crear la pagina de manera rápida e incorporando con un framework la unidad de persistencia. Pero como desventaja esta el concepto de no tener elasticidad para hacer otras cosas. Parámetro por el cual Ruby on Rails podría llevar a cabo de mejor manera. Pero el punto frágil de Ruby es que como utiliza un Framework de propósito general (igual que PHP) su capacidad de aprendizaje es bastante costoso. 
\section{Fundamentos Teóricos}
\subsection{Tecnologías técnicas que se ocuparán}
\begin{enumerate}
\item Servidor de Aplicaciones Web: Ya sea apache Tomcat, JBoss o Glasfish, en general un servidor de aplicaciones nos permite tomar un servidor y convertirlo en un ejecutante sobre la máquina virtual de Java. Lo que ejecutaremos serán códigos compilados en Java y páginas. Dando lugar a proyectos Web.
\item Ant: Creador de proyectos Java, permite la creación de tareas de compilación que son las encargadas de crear el programa que se inserta sobre el servidor de aplicaciones web de apache Tomcat/JBoss/Glasfish.
\item Maven: Creador de proyectos en Java, permite la integración de la compilación de las clases con los archivos del proyecto entero, manipula las dependencias y ahorra el tedioso trabajo de ordenamiento del proyecto, todo de acuerdo a estandartes mundiales de creación para proyectos Java.
\item JPA: Implementación para la abstracción de un framework de programación sobre el modelo de datos y la respectiva base de datos, sobre spring.
\item Unidad de Encriptamiento: Sistema que nos permite manejar los mensajes encriptados desde el emisor hasta el receptor. Incluso con otros sistemas frente a ciertas condiciones y restricciones.
\end{enumerate}
\subsection{Frameworks que se ocuparán}
\begin{enumerate}
\item JSF: Poderosa herramienta que permite la programación de la interacción entre la vista y el controlador del proyecto, incorpora una serie de herramientas útiles que facilitan el desarrollo del controlador de forma limpia y eficiente. También se incluye una serie de elementos de la vista que facilitan la forma de mostrar la información al usuario final.
\item PrimeFaces: Conjunto de librerías que proporcionan una visual y una estética para mostrar el dinamismo de las vistas de forma elegante y profesional. Facilita el desarrollo de las vistas incorporando a-sincronismo en varias de sus funcionalidades. 
\item Spring: Framework Java que permite la integración de modelos de programación que se hacer fuertemente útiles para el modelo del proyecto, incorpora inyección de dependencias que ordena la forma de programar, sin tener que ensuciar código, estableciendo también AOP que es un paradigma de programación enfocado en el aspecto que deben tener las clases del controlador y como éstas van siendo inyectadas para su potencial uso. 
\item MyBatis: Implementación de JPA que permite la correcta abstracción del modelo de bases de datos hacia las clases del modelo del proyecto.
\item PrettyFaces: Framework que nos permite mejorar la forma de implementar la navegación dentro del sistema de páginas. Mas que nada usado para contener las excepciones de error y la redirección a una pagina de error en el caso de la ejecución haya salido errónea.
\end{enumerate}
\subsection{Forma genérica de la Solución}
El proyecto se entregará primeramente a ciertas fechas de entrega, en cumplimiento de paquetes de requerimientos y funcionalidades previamente conversadas. Para este caso es posible que dichas entregas tengan ciertos cambios previstos por el profesor a cargo del ramo. Por tanto es importante destacar que las directrices sobre entregas las tiene el encargado del ramo.
\subsection{Implementación de la Solución}
La aplicación será implementada una vez se haga un estudio de verificación de los requerimientos, para tal caso se medirá todas las variables involucradas en el proceso de implementación y una vez que el proyecto pase por todas esas pruebas, finalmente será apto para su producción y posterior utilización.
\section{Ciclo de vida del Software}
\begin{enumerate}
\item Toma de Requerimientos: Proceso que básicamente comprende 2 actividades, primero la toma de requerimientos en base al sistema SEPA anterior, segundo una serie de entrevistas para captar las historias de usuarios que posteriormente se convierten en requerimientos.
\item Análisis de Requerimientos: Etapa en donde el profesor encargado discute con el alumno cuales requerimientos deben realizarse, de acuerdo con las expectativas y las necesidades del equipo de trabajo.
\item Diseño de datos: Proceso en donde se desarrolla el modelo de datos que debe soportar a los requerimientos, cada parte del modelo corresponde a un grupo de requerimientos aceptados como una ramificación completa del sistema a desarrollar.
\item Diseño Estructural: En este punto el alumno trabaja desarrollando el proyecto en coordinación con el profesor encargado y este va revisando constantemente el avance, emitiendo preferencias y aclaraciones.
\item Pruebas: Proceso por el cual el sistema en una fase anterior al termino de proyecto se encuentra en estudio, el profesor encargado verifica la fiabilidad y calidad de los requerimientos ya resueltos.
\item Instalación: En este punto se produce la incorporación del sistema en su ambiente por defecto, lo que implica un termino del proyecto, solo dentro del marco de trabajo de título para el alumno de este proyecto.
\end{enumerate}
\section{Historia del Arte}
La cantidad de datos que se obtienen en un establecimiento universitario, es de vital importancia para el completo desarrollo de sus actividades, ya que permiten una normal toma de decisiones que afectan positivamente el crecimiento de la institución académica, ya sea de forma docente como administrativa. 

Una de las mas importantes finalidades que contempla un estudio estadístico, responde a la necesidad de los estudiantes, las familias y los profesores por elegir el mejor plan de estudios que el mercado puede ofrecerles, de esta manera, es necesario señalar que todas las entidades educacionales debieran tener la misión de informar acerca de su situación en el grupo de universidades al cual representa.

De igual manera, los estudios estadísticos son importantes para el desarrollo de la entidad educacional, ya que responden a la necesidad propia de la universidad de ver como se encuentra dentro del mercado. En pos de mejorar las condiciones y estatutos actuales. Buscando como objetivo llegar mas cerca de su visión como organización.

En chile esta creciendo la necesidad de que las instituciones deben estar evaluadas por organismos preparados para testificar las características de su desempeño y los profesionales que produce. Para estos efectos, existen organismos estatales que acreditan lo que cada universidad propone, pero en otro lado existen entandares de calidad que son medidos por organismos extranjeros, para dar un sello de calidad, evaluado por datos estadísticos\cite{data1}.

En 2009 se abre el sitio QS Top Universities que hasta el día de hoy es un sitio altamente eficiente de evaluación a trabes de ránkings enfocados a entregar información a gente proveniente de post-grados, MBA, estudios de ingeniería y negocios. Desarrollando estadísticas de alto alcance en sitios web, eventos, guias y programas por Internet y finalmente entregando soluciones técnicas a otros organismos, tales como universidades o empresas privadas, entre otras.

Sus datos manejan información de 41 países, entre los que están: Argentina, Ecuador, Japon, Singapur, Australia, Egipto. Su servicio esta libre para todo el publico y permite un tipo de suscripción gratuita para visualizar tablas comparativas de manera bien simple.

Permite hacer comparaciones por categorías varias, ya sea país, estado, ciudad y otras características internas de la universidad como ponderaciones o prestaciones de alumnos, tales como cantidad de alumnos o la segregación por sexo o etnia en cada localidad.

Su objetivo es ser el sitio principal y mundial de ránkings de educación para los profesionales ambiciosos que buscan fomentar tanto su desarrollo personal y profesional. Acentuando sus características en entregar soluciones on-line de desarrollo y visualización con alcances estadísticos y comparativos a quienes estén dentro de su sistema.

Es una empresa de tamaño mediano, con más de 150 empleados en oficinas de todo el mundo. También se realizan Tours en 70 ciudades en 39 países que permiten a más de 50.000 espectadores, los cuales buscan entrar a una universidad\cite{data2}.

El Ránking de Calidad de las Universidades Chilenas, es un trabajo realizado por AméricaEconomía Intelligence en el 2010, un sitio de ránking para instituciones educacionales en Chile, que utiliza ciertos criterios, como lo son la cantidad de investigaciones, o la masa de estudiantes que entran en cada universidad.

Algunos datos generales que se obtienen, son que la Universidad de Chile y la PUC, tienen diferencias muy pequeñas en el campo de la investigación, para lo cual se mostró que la Universidad de Chile aporto con 1.329 papers publicados, en cambio la PUC hizo un aporte total de 1.029 publicaciones.

Mucha de esta información estadística es recopilada a trabes de encuestas enviadas a distintas universidades y organismos de educación en chile. De esta manera se puede obtener información importante para ser mostrada al publico que desea tomar una decisión en el mercado.

Este sistema da cuenta de una necesidad por parte de las universidades, de ser mas competitiva en el mercado universitario. Aumentando los esfuerzos por adquirir más estudiantes, mostrando mejores profesores, alumnos e investigación.

Finalmente se nota a las universidades actuales buscando organismos extranjeros capaces de certificar las características de la educación que ofrecen, de esta manera muestran no solo una acreditación por un organismo chileno, si no también una necesidad por competir a nivel internacional\cite{data3}.

\bibliographystyle{plain}
\bibliography{bibliografia.bib}
\end{document}