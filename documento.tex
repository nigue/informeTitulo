%definición del artículo
\documentclass[a4paper,12pt,openany,oneside]{book}
\usepackage[left=5cm,right=2cm,top=4cm,bottom=4cm,paperwidth=216mm,paperheight=330mm,pdftex]{geometry}
%paquete usado para silabación en español
\usepackage[spanish]{babel}
%codificación del documento
\usepackage[utf8]{inputenc}
%espaciado
\linespread{1.5}
%identación de párrafo
\setlength{\parindent}{20pt}
%espaciado de párrafo
\setlength{\parskip}{4ex plus 0.5ex minus 0.2ex}
%para validar sólo sintaxis sin compilar
%\usepackage{syntonly}
%\syntaxonly
%para usar imágenes
\usepackage{graphicx}
%para usar fragmentos de codigo fuente
\usepackage{listings}
\usepackage{float}
\floatstyle{boxed}
\newfloat{codigo}{thp}{lop}
\floatname{codigo}{Caja de Código}
%comienzo del documento
\begin{document}
\thispagestyle{empty}
\begin{center}
\textbf{UNIVERSIDAD TECNOLÓGICA METROPOLITANA\\
ESCUELA DE INFORMÁTICA}\\
\vspace{3cm}
SISTEMA ESTADISTICO DE PROFESORES Y ALUMNOS\\S.E.P.A.
\end{center}
\begin{flushright}
TRABAJO DE TÍTULO PARA OPTAR AL\\
TÍTULO DE INGENÍERO CIVIL EN\\
COMPUTACIÓN\\
MENCIÓN INFORMÁTICA.\\
\vspace{3cm}
PROFESOR GUÍA: Sebastián Salazar Molina\\
\vspace{1.5cm}
Miguel Angel Aníbal Davor Fuenzalida Pino
\end{flushright}
\vspace{4cm}
\begin{center}
SANTIAGO - 2013
\end{center}
\newpage
\thispagestyle{empty}
\begin{flushright}
\vspace{20mm}
Nota: \line(1, 0){140} \\
\vspace{30 mm}
\line(1, 0){180}\\	
Firma y Timbre\\
Autoridad Responsable
\end{flushright}
\chapter*{Resumen}
\thispagestyle{empty}
Este trabajo consiste en la reestructuración del proyecto CEPA, enfocado al control de información, perteneciente a la Universidad Tecnologica Metropolitana. A traves de tecnologías potentes que nos permiten un diseño moderno y flexible, a su vez nos facilitan el proceso de creación de nuevas entidades y caracteristicas. Permitiendo alargar la vida de este proyeto. Cabe destacar que el proyecto se hará bajo la metodologia PSP que nos proporciona un modelo de trabajo basado en la caracterización de los tiempos y costes del trabajo de un Ingeniero Informatico. De esta manera, este proyecto es una nueva forma de potenciar las capacidades operativas de la Universidad, manejando de mejor manera su calidad y competividad dentro del mercado de Universidades.

Para hacer este proyecto se establece un modelo de trabajo dado por la metodología PSP (Personal Software Process) y adicionando documentacón extra que esta en el apartado Anexos, la cual responde a eventos que caracterizaron las directrices de como se lleva adelante este trabajo. Tambien es importante destacar que como es un trabajo de renovación, este proyecto tendrá una nueva forma y permitirá a más usuarios poder aceder a información de la universidad.

Finalmente este proyecto esta enfocado en ser un ejemplo para potenciar las capacidades administrativas de la Universidad, haciendola un ente mas proactivo a mejorar. Por la calidad de la educación y la exelencia en esta tarea.
\tableofcontents
\listoffigures
\chapter*{Introducción}
\thispagestyle{empty}
La información es un recurso sumamente importante en las organizaciones. Esta permite hace la toma de decisiones y la manipulación de los recursos físicos. Pero por si sola no es posible hacer predicciones, por lo que los usuarios de estos datos necesitan datos procesados de manera amena, en pos de tener valores cuantificables en el entorno en donde se trabaja. La cantidad de información puede crecer y a medida que pasa el tiempo y el tamaño, también aumenta la complejidad de sostener la información. Lo que sugiere 2 cosas, entre mas información es mas probable que los valores finales de los datos procesados sean mayormente fidedignos, con mayores volúmenes de información se espera tener sistemas de manejo de mayor envergadura y capacidad. Razón por la cual la información se ha vuelto un recurso tan necesario e importante en todo tipo de organización, ya sea una empresa publica como una universidad o una empresa privada.

Hoy en día la Universidad Tecnológica Metropolitana esta en una situación en donde tiene muchos proyectos, de los cuales una poca cantidad están en real uso, de los que se efectivamente se encuentran uso actual, exsisten sistemas altamente capaces de procesar volúmenes de información, que particularmente entregan indicadores de alta confianza. El sistema que más indicadores usa es el CEPA, el cual permite a la universidad establecer directrices con respecto a varios indicadores que arroja el sistema. Este proyecto fue realizado por Sebastían Salazar Molina para su Trabajo de Título y corresponde a lo que la universidad tiene para medir su calidad a través de información procesada. Actualemtne el Sistema no tiene conexión con la información proveniente de otro sistema llamado DirDoc, el cual contiene las actas de notas, para los estudiantes de la Universidad. Esta característica faltante está dentro de un compendio de necesidades que deben ser satisfechas en pos de ayudar a la situación actual de la Universidad en estos periodos donde necesita mostrar su calidad y su competitividad frente a otras universidades.

En este proyecto se propone realizar una renovación total del sistema CEPA incluyendo mejoras que puedan salir de la toma de requerimientos, siendo parte de un conjunto de cosas que deben existir como mínimo, agregando también requerimientos que puedan ser útiles para hoy en día. También incluir la capacidad de que pueda ser mejorado en el corto plazo en el futuro.

Para cumplir este propósito se propone la estructuración del código con java, lo que nos permite hacer uso de herramientas que nos facilitarán el desarrollo y el diseño del proyecto. Es importante destacar que ocuparemos tecnologías que básicamente son recientes, y que nos permiten diseñar aplicaciones con el estándar Modelo Vista Controlador de manera rápida y ágil, para este propósito existen herramientas que nos facilitarán la comunicación entre las distintas capas del proyecto, facilitando aun mas la puesta en marcha de este proyecto.

Para el desarrollo integro de este proyecto se hará uso de una metodología de desarrollo de proyectos de manera individual, perfecto para trabajos de Título como el que se está realizando. La metodología PSP (Personal Software Process) establece aristas de desempeño que permiten a su usuario responder a patrones de trabajo que futuramente serán la base para calcular costes y tiempos que un ingeniero necesita para la realización de sus actividades de manera rápida y vital.

Existen 3 tipos de beneficios que una organizacion puede obtener al hacerse con un sistema moderno de captura integra de modelos de datos. El primero corresponde al beneficio interno que se produce, este conlleva una importante evolución de como se ve la organizacion por dentro. El segundo corresponde al beneficio de los usuarios de estos sistemas modernos, los cuales se ven favorecidos, ya que tendrán a su disposición una completa herramienta que los apoyará en sus actividades, de la mejor manera posible. Finalmente el beneficio a futuro, para quienes se den la tarea de complementar este sistema o mejorarlo, ya que no les va a ser mas difícil la implementación de mejoras y cambios.

Para este proyecto en particular se espera una pronta puesta en marcha, ya que es bastante necesario para la Universidad como para los usuarios finales de este sistema. Esto apoya el concepto de estar con una mirada innovadora frente a proyectos que potencian las capacidades de generar prontas respuestas dentro de la organización. Para los usuarios, el sistema va a ser un complemento de enorme ayuda, de manera rápida, segura y fácil.

En el Capitulo 1 se tratan los antecedentes generales que hacen del proyecto una nececidad hacia la universidad y como estas nececidades se transforman en problemas que deven tener una solución concreta y rápida, con el fin de cumplir ciertos logros academicos dentro de la organización.

En el Capitulo 2 se puede ver las caracteristicas principales de la organización y como se encuentra a la hora de recibir un proyecto que potenciará su capacidad de evaluación de exelecia y calidad.

En el Capitulo 3 se muestran las caracteristicas principales del problema y como este deve ser solucinado, de que manera nosotros como estudiantes podemos dar una solución logica que imprima calidad a los procesos productivos de la universidad.

En el Capitulo 4 se explaya en los limites de la solución, la cual está en el marco de un proyecto de trabajo de título, lo que quiere decir que hay ciertas restricciones y alcances a la hora de abordar este tipo de proyectos.

En el Capitulo 5 se relata un poco de historia de las herramientas que permiten la pronta realizacion de las actividades de desarrollo. 
\chapter{Antecedentes}
\thispagestyle{empty}
\section{Contexto General}
El proyecto que este informe de Título expone esta inmerso sobre la Universidad Tecnológica Metropolitana (UTEM), la cual es un organismo de enseñanza superior que se encuentra dentro del Consejo de Rectores de las Universidades Chilenas. Su caracteristica principal es ser una de las primeras en inmersión de la primera generación de estudiantes universitarios para las familias de clase media, en compraración a otras universidades estatales.

La importancia de esta universidad radica en el echo de que tiene como misión la formación de profesionales que sean lo mas competitivos en el mundo laboral. Mostrando en sus profesionales una calidad competitivamente fuerte frente a otras universidades caracterizadas por estudiantes provenientes de familas que ya poseen profesionales dentro de su nuecleo.

Esto amerita medidas de certificacion de calidad, para la organización en general, por lo que se nesesita claramente la acreditación de la universidad y sus carreras.
\section{Contexto Particular}
Actualmente la universidad se encuentra en un nuevo proceso de acreditación, lo que responde a una nececidad de poseer herramientas que permitan la evaluacion de la calidad docente y administrativa de la universidad.

El proyecto SEPA, Sistema Estadístico de Profesores y Alumnos, es el encargado de mostrar información que pueda ser cuantizable y claramente medible en pos de tener herramientas de control general y toma de deciciones.
\section{Problema}
Para la nueva acreditación (de la comisión acreditadora) que se encuentra en preparación, ya a comienzos del año 2013, se nesesita la reingenieria del proyecto SEPA, el cual sera una versión renovada que mostrará mas caracteristicas y permitirá una mejor forma de evaluar la calidad docente y administrativa de la universidad.

El problema en este caso es la realización de estas ectividades a favor de los procesos administrativos que requieren los organismos que evaluen y acrediten a la Universidad Tecnológica Metropolitana. De esta manera se pone en discusión la pronta nesecidad de tener una herramienta que repesente una versión mejorada de el sistema anterior SEPA.
\section{Motivación}
Uno de los primeros motivos que aparece, es el echo de ayudar a la Universidad, con un proyecto que potencie las capacidades de esta. Convirtiendola en un organismo mas competitivo en el mercado e impulsando el desarrollo de sus capacidades. Esperando que con el tiempo se de la posibilidad de crear nuevas cosas a partir del paso que este proyecto puede generar, esto quiere decir que la Universidad debe estar mas abierta a recibir proyectos internos, que sean hechos por sus propios alumnos, quienes son parte fundamental del desarrollo estudiantil y conocen de mejor manera las cualidades y características de su Universidad. Este proyecto tiene como motivación generar una herramienta de apoyo que sea de calidad y que beneficie los procesos productivos.

La motivación de parte del alumno comprende el aprendizaje de tecnologías, metodologías y herramientas que le permiten en un futuro tener mejores oportunidades y ser mas competitivo en el ámbito laboral. Pues esto supone la composición de un sistema el cual requiere dominar ciertas tecnologías de manera acabada. Esta motivación es importante ya que no solo es una experiencia de aprendizaje, si no que también supone una manera de ver desafíos propios, tales como proyectos propios que el alumno pueda tener en algún futuro. Pensando de esta manera es lógico suponer el echo de que el alumno tendrá la necesidad de realizar proyectos en pos de mejorar las condiciones de organizaciones benéficas. En resumen el ayudar es una motivación grande.
\chapter{La Organización}
\thispagestyle{empty}
\section{Misión}

\section{Visión}

\section{Organigrama Empresarial}

\section{Estructura de desarrollo el proyecto}

\chapter{Problema}
\thispagestyle{empty}
\section{Diagnóstico sin proyecto}

\section{Diagnóstico con proyecto}

\section{Objetivos Generales y Específicos}
\subsection{Objetivos Generales}
\textit{Dessarrollar un sistema de medición estadistica para la Universidad Tecnológica Metropolitana ayudando a sus procesos docentes y productivos, mejorando la calidad de la educación y los niveles de enseñanza academica que la organización desea potenciar.}
\begin{enumerate}
\item Un objetivo general es la creación de un sistema web basado en el estándar Modelo Vista Controlador, asegurando un producto que sea eficiente en el cumplimiento de sus objetivos. Para esto se espera que el sistema tenga a lo menos un grado de sustentabilidad que lo haga eficiente, rápido y cómodo a la hora de usar, para todos los tipos de usuarios que se establecerán. Las características principales serán que este sistema estará echo en Java lo que nos permite una fácil implementación de tecnologías, las cuales nos ayudarán en el proceso de construcción de un proyecto en el marco de las TIC, especialmente el de la creación de proyectos web.
\item El otro objetivo principal es la utilización de una metodología de desarrollo, en este caso PSP (Personal Software Process), la cual nos permite llevar un seguimiento de las actividades que se deben realizar, de manera lógica y ordenada, para este propósito la metodología propone la utilización de valores contables en el tiempo, estos valores establecen las características y condiciones del trabajo del ingeniero. Por lo cual se tendrá en cuenta todo un estudio con respecto al trabajo que se realiza y seŕa parte integral del proyecto de Trabajo de Título.
\end{enumerate}
\subsection{Objetivos Específicos}
\begin{enumerate}
	\item Desarrollar un sistema que sea eficiente para todos los tipos de usuario.
	\item Facilitar la creación o la producción de mejoras sobre este nuevo sistema.
	\item Mostrar el desarrollo del proyecto a través de una metodología de desarrollo como lo es PSP.
	\item Potenciar las capacidades de la universidad, entregando un producto de calidad para una organización enfocada en la producción de profesionales del futuro.
\end{enumerate}
\chapter{Alcances y Limitaciones}
\thispagestyle{empty}
\section{Alcances del Proyecto}
El primer compromiso, de carácter general, que se establece para este proyecto, es la reestructuración del proyecto CEPA en el lenguaje Java. Esto indica la necesidad de ocupar tecnologías que rodean al standard de diseño llamado J2EE, el cual nos proporciona herramientas que facilitan el desarrollo de aplicaciones MVC (Modelo Vista Controlador) a través de sus componentes. A esto hay que agregar el echo de que se tendrán en cuenta otras tecnologías que nos facilitan el manejo de los datos, como lo es Spring, que resulta importante en la creación de proyectos Web basados en Java.

Como segundo alcance, el proyecto estará enfocado en la coparticipación de indicadores UTIGRA (Unidad de Títulos y Grados), esto quiere decir que de aquí en adelante también trabajaremos con información pertinente a gente que ha terminado sus procesos estudiantiles, mostrándonos un nuevo abanico de información. De esta manera terminamos con una arista importantísima a la hora de establecer un marco de trabajo.

Para el proyecto en particular se tendrán en cuenta la creación de una nueva cantidad de perfiles, adaptandose a las nuevas necesidades y sobre todo al cambio con relación a UTIGRA. Para esto se incorporan mas perfiles de los que se tenían en un principio, aumentando la cantidad de usuarios del sistema y ampliando sus posibilidades.
\section{Limitaciones del Proyecto}
Por otro lado, cabe mencionar que el proyecto no contendrá requerimientos que se hagan fuera de plazo, esto es importante ya que como el trabajo que se realiza es parte del desarrollo que se ve sustentado a través de una metodología de desarrollo como lo es PSP, no es conveniente agregar mas requerimientos. Esto no quiere decir que la metodología no acepte la inclusión de requerimentos a mitad de trabajo, pero es necesario establecer esa política en el marco de un trabajo de Título.
\section{Alcances del Trabajo de Título}

\section{Limitaciones del Trabajo de Título}

\chapter{Marco Teórico}
\thispagestyle{empty}
\section{Base Conceptual}
\subsection{Establecer el problema}

\subsection{Establecer la Solución}

\subsection{¿Por que ocupar Java?}

\subsection{¿Por que no otras tecnologías?}
En comparación con PHP el cual nos permite crear la pagina de manera rápida e incorporando con un framework la unidad de persistencia. Pero como desventaja esta el concepto de no tener elasticidad para hacer otras cosas. Parámetro por el cual Ruby on Rails podría llevar a cabo de mejor manera. Pero el punto frágil de Ruby es que como utiliza un Framework de propósito general (igual que PHP) su capacidad de aprendizaje es bastante costoso. 
\section{Fundamentos Teóricos}
\subsection{Tecnologías técnicas que se ocuparán}
\begin{enumerate}
\item Servidor de Aplicaciones Web: Ya sea apache Tomcat, JBoss o Glasfish, en general un servidor de aplicaciones nos permite tomar un servidor y convertirlo en un ejecutante sobre la maquina virtual de Java. Lo que ejecutaremos serán códigos compilados en Java y paginas. Dando lugar a proyectos Web.
\item Ant: Creador de proyectos Java, permite la creación de tareas de compilación que son las encargadas de crear el programa que se inserta sobre el servidor de aplicaciones web de apache Tomcat/JBoss/Glasfish.
\item Maven: Creador de proyectos en Java, permite la integración de la compilación de las clases con los archivos del proyecto entero, manipula las dependencias y ahorra el tedioso trabajo de ordenamiento del proyecto, todo de acuerdo a estandartes mundiales de creación para proyectos Java.
\item JPA: Implementación para la abstracción de un framework de programación sobre el modelo de datos y la respectiva base de datos, sobre spring.
\item Unidad de Encriptamiento: Sistema que nos permite manejar los mensajes encriptados desde el emisor hasta el receptor. Incluso con otros sistemas frente a ciertas condiciones y restricciones.
\end{enumerate}
\subsection{Frameworks que se ocuparán}
\begin{enumerate}
\item JSF: Poderosa herramienta que permite la programación de la interacción entre la vista y el controlador del proyecto, incorpora una serie de herramientas útiles que facilitan el desarrollo del controlador de forma limpia y eficiente. También se incluye una serie de elementos de la vista que facilitan la forma de mostrar la información al usuario final.
\item PrimeFaces: Conjunto de librerías que proporcionan una visual y una estética para mostrar el dinamismo de las vistas de forma elegante y profesional. Facilita el desarrollo de las vistas incorporando a-sincronismo en varias de sus funcionalidades. 
\item Spring: Framework Java que permite la integración de modelos de programación que se hacer fuertemente útiles para el modelo del proyecto, incorpora inyección de dependencias que ordena la forma de programar, sin tener que ensuciar código, estableciendo también AOP que es un paradigma de programación enfocado en el aspecto que devén tener las clases del controlador y como estas van siendo inyectadas para su potencial uso. 
\item Hibernate: Implementación de JPA que permite la correcta abstracción del modelo de bases de datos hacia las clases del modelo del proyecto.
\item PrettyFaces: Framework que nos permite mejorar la forma de implementar la navegación dentro del sistema de paginas. Mas que nada usado para contener las excepciones de error y la redirección a una pagina de error en el caso de la ejecución haya salido errónea.
\end{enumerate}
\subsection{Forma genérica de la Solución}

\subsection{Implementación de la Solución}

\end{document}