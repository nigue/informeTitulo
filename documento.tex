%definición del artículo
\documentclass[a4paper,12pt,openany,oneside]{book}
\usepackage[left=5cm,right=2cm,top=4cm,bottom=4cm,paperwidth=216mm,paperheight=330mm,pdftex]{geometry}
%paquete usado para silabación en español
\usepackage[spanish]{babel}
%codificación del documento
\usepackage[utf8]{inputenc}
%espaciado
\linespread{1.5}
%identación de párrafo
\setlength{\parindent}{20pt}
%espaciado de párrafo
\setlength{\parskip}{4ex plus 0.5ex minus 0.2ex}
%para validar sólo sintaxis sin compilar
%\usepackage{syntonly}
%\syntaxonly
%para usar imágenes
\usepackage{graphicx}
%para usar fragmentos de codigo fuente
\usepackage{listings}
\usepackage{float}
%para usar signos de check y wrong
\usepackage{bbding}
%para usar colores en textos y simbolos
\usepackage{color}
\floatstyle{boxed}
\newfloat{codigo}{thp}{lop}
\floatname{codigo}{Caja de Código}
%comienzo del documento
\begin{document}
\thispagestyle{empty}
\begin{center}
\textbf{UNIVERSIDAD TECNOLÓGICA METROPOLITANA\\
ESCUELA DE INFORMÁTICA}\\
\vspace{3cm}
SISTEMA ESTADÍSTICO DE PROFESORES Y ALUMNOS\\S.E.P.A.
\end{center}
\begin{flushright}
TRABAJO DE TÍTULO PARA OPTAR AL\\
TÍTULO DE INGENIERO CIVIL EN\\
COMPUTACIÓN\\
MENCIÓN INFORMÁTICA.\\
\vspace{3cm}
PROFESOR GUÍA: Sebastián Salazar Molina\\
\vspace{1.5cm}
Miguel Ángel Aníbal Davor Fuenzalida Pino
\end{flushright}
\vspace{4cm}
\begin{center}
SANTIAGO - 2013
\end{center}
\newpage
\thispagestyle{empty}
\begin{flushright}
\vspace{20mm}
Nota: \line(1, 0){140} \\
\vspace{30 mm}
\line(1, 0){180}\\	
Firma y Timbre\\
Autoridad Responsable
\end{flushright}
\chapter*{Resumen}
\thispagestyle{empty}
Para la toma de decisiones y la manipulación de recursos de las organizaciones se requiere de información procesada.

La UTEM dispone actualmente, para la administración de su información el proyecto SEPA, que consiste en un proyecto Web realizado en PHP que toma datos de alumnos, profesores y encargados, para posteriormente hacer cálculos predefinidos con el fin de mostrar indicadores que representen la realidad universitaria.

En su actual desempeño, se ha visto que el SEPA, presenta insuficiencias:

% ejemplos, no tiene servicio web dirdoc, inconsistencia de datos, carece de formas alternativas de accesos, carece multiples roles (más de 2).

El presente trabajo de Título representa una mejora al sistema SEPA, porque enfrenta de mejor forma la diversidad de formatos de datos empleando nuevas herramientas y aplicaciones, tales como: 

% Java, plr, , d

\tableofcontents
\listoffigures
\chapter*{Introducción}
\thispagestyle{empty}
La Universidad Tecnológica Metropolitana utiliza para la administración de toda la información referente a estudiantes, profesores y personal de administración un sistema denominado proyecto SEPA que se basa en la tecnológica PHP. Este proyecto constituyó una forma de potenciar las capacidades operativas de la Universidad, manejando de mejor manera su calidad y entregando una ventaja comparativa frente a otras universidades.

El objetivo de este Trabajo de Título consiste en la actualización del proyecto SEPA mediante la tecnología Java que utiliza herramientas más estructuradas y conocidas por programadores y que tienen como resultado la materialización de sistemas informáticos más fácilmente perfectibles. Por ejemplo, se utilizará el Modelo Vista Controlador de manera rápida y ágil mediante el uso de JPA (para establecer el modelo de base de datos), Spring (para facilitar el desarrollo de la lógica del negocio) y PrimeFaces (para mostrar una visualización cómoda para los usuarios). 

Además, el proyecto se hará bajo la metodología PSP (Personal Software Process) que nos proporciona un modelo de trabajo basado en la caracterización de los tiempos y costes del trabajo de un ingeniero informático, adicionando documentación extra que está en el apartado Anexos, el cual responde a eventos que caracterizaron las directrices de cómo se lleva adelante este trabajo, todo lo cual, conduce a una evaluación de la calidad del proyecto como programa.

Existen 3 tipos de beneficios que una organización puede obtener al hacerse con la versión mejorada del sistema SEPA:

\begin{enumerate}
\item Beneficio interno que se produce cuando se ve la organización por dentro y se mostrará a la universidad como un ente preocupado de medir de manera cuantificable su calidad como organización.

\item Beneficio de los usuarios, los cuales se ven favorecidos, ya que tendrán a su disposición una completa herramienta que los apoyará en sus actividades.

\item Beneficio a futuro, dado el alto estándar de las herramientas utilizadas, la integración, debiese resultar más sencilla, ya que este proyecto esta realizado bajo tecnologías más estandarizadas.
\end{enumerate}

\chapter{Antecedentes}
\thispagestyle{empty}
\section{Contexto General}
El proyecto que este informe de Título expone esta inmerso en la Universidad Tecnológica Metropolitana (UTEM), la cual es un organismo de enseñanza superior que se encuentra dentro del Consejo de Rectores de las Universidades Chilenas.
\section{Contexto Particular}
El proyecto SEPA, Sistema Estadístico de Profesores y Alumnos, es el encargado de mostrar información que pueda ser cuantificable en pos de tener herramientas de control general y toma de decisiones.
\section{Problema}
% INSUFICIENCIAS
\section{Motivación}
Ayudar a mejorar la administración de la Universidad y aprender a desarrollar tecnologías que apoyan a instituciones publicas que ayudan a la gente.
\chapter{La Organización}
\thispagestyle{empty}
\section{Misión}
La Universidad Tecnológica Metropolitana es una organización estatal docente enfocada en la generación de profesionales que estén a la altura de los proyectos que se necesitan en pos de mejorar el país de manera sólida, segura y tecnológica. Para este motivo transforma a personas en profesionales con altas capacidades para enfrentar problemas. Estos profesionales se caracterizan por la fuerte convicción de lograr sus objetivos incluso en situaciones difíciles.
\section{Visión}
La organización será un ente de enseñanza que contará con una inclusión de profesionales de un alto nivel académico, mejorando así su capacidad de formar los profesionales del futuro. Tendrá áreas de investigación y creación tecnológica que permitirán la modernización en muchas áreas del conocimiento. La organización se caracterizará por la prescindencia de todo tipo de discriminación de genero, religión y raza.
\section{Organigrama Empresarial}

\section{Estructura de desarrollo el proyecto}
El desarrollo del proyecto esta bajo la metodología de desarrollo Personal Sofware Process (PSP), la cual permite medir en forma numérica todas la variables implicadas en el proceso productivo de la creación de este proyecto, lo que será recompensado con valiosa información tanto del proyecto como del alumno.
\chapter{Problema}
\thispagestyle{empty}
\section{Situación Actual}
La organización dispone del proyecto anterior llamado SEPA, para estos efectos dispone de una paleta de funcionalidades, que esperan en un futuro poder mejorar. Mientras tanto se pueden generar gráficos y variables estadísticas que muestran información útil a la hora de tomar decisiones. Para estos efectos ya se han pedido mejoras e inclusión de más actores frente a este sistema.
\section{Situación Propuesta}
La universidad dispone de un sistema de análisis estadístico que permite la toma de decisiones de manera más especializada, enfocada a los distintos tipos de usuario que pueden existir en el conjunto de la organización.
\section{Objetivos Generales y Específicos}
\subsection{Objetivos Generales}
\textit{Desarrollar un sistema de medición estadística para la Universidad Tecnológica Metropolitana ayudando a sus procesos docentes y productivos, mejorando la calidad de la educación y los niveles de enseñanza académica que la organización desea potenciar.}
\begin{enumerate}
\item Crear un sistema web basado en el estándar Modelo Vista Controlador, asegurando un producto que sea eficiente en el cumplimiento de sus objetivos. Para esto se espera que el sistema tenga a lo menos un grado de sustentabilidad que lo haga eficiente, rápido y cómodo a la hora de usar, para todos los tipos de usuarios que se establecerán. Las características principales serán que este sistema estará hecho en Java lo que nos permite una fácil implementación de tecnologías, las cuales nos ayudarán en el proceso de reconstrucción de un proyecto en el marco de las TIC, especialmente el de la creación de proyectos web.
\item El otro objetivo principal es la utilización de una metodología de desarrollo, en este caso PSP (Personal Software Process), la cual nos permite llevar un seguimiento de las actividades que se deben realizar, de manera lógica y ordenada, para este propósito la metodología propone la utilización de valores contables en el tiempo, estos valores establecen las características y condiciones del trabajo del ingeniero. Por lo cual, se tendrá en cuenta todo un estudio con respecto al trabajo que se realiza y será parte integral del proyecto de Trabajo de Título.
\end{enumerate}
\subsection{Objetivos Específicos}
\begin{enumerate}
	\item Desarrollar un sistema que sea eficiente para todos los tipos de usuario.
	\item Facilitar la creación o la producción de mejoras sobre este nuevo sistema.
	\item Mostrar el desarrollo del proyecto a través de una metodología de desarrollo como lo es PSP.
	\item Potenciar las capacidades de la universidad, entregando un producto de calidad para una organización enfocada en la producción de profesionales del futuro.
\end{enumerate}
\section{Esquema de trabajo}
El alumno y el profesor se juntan una vez a la semana para discutir lo que se va a realizar hasta la próxima, arreglando errores y agregando más peticiones.

Las peticiones del profesor son actividades que serán estudiadas bajo la metodología PSP.
\chapter{Alcances y Limitaciones}
\thispagestyle{empty}
\section{Alcances del Proyecto}
La coparticipación de indicadores UTIGRA (Unidad de Títulos y Grados) para trabajar con información pertinente a gente que ha terminado sus procesos estudiantiles en la UTEM.

Se crearan una serie de nuevos perfiles, y cada uno de estos tendrán su nivel de alcance en el sistema.
\section{Limitaciones del Proyecto}
El proyecto no contendrá requerimientos que se hagan fuera de lo requerido por el profesor como actividades de Trabajo de Título.
\section{Alcances del Trabajo de Título}
Los alcances del Trabajo de Título están delimitados por las características que el trabajo de título debe tener, es decir, que se hace un marco de trabajo, delimitado por el profesor, que comprende los tópicos a trabajar.
\section{Limitaciones del Trabajo de Título}
El trabajo no comprende áreas que estén fuera de los límites temporales a la toma de requerimientos, por lo cual es necesario indicar que hay características y funcionalidades que a pesar de ser levantadas puede que no sean parte de este trabajo de título.
\chapter{Marco Teórico}
\thispagestyle{empty}
\section{Base Conceptual}
\subsection{Establecer el problema}
En la universidad existe la necesidad de mostrar información relevante para la toma de decisiones, esta información debe ser respaldada por las grandes cantidades de registros que se guardan constantemente en la universidad. Para este propósito existe un sistema llamado Sistema Estadístico de Profesores y Alumnos (SEPA), el cual otorga información importante para la toma de decisiones y verificar la calidad de la educación. Pero como la universidad ha crecido, también crece la cantidad de información y la necesidad de tener datos más fuertes y de manera más rápida. Desde este punto se plantea la necesidad de renovar el sistema SEPA, para cumplir con las exigencias que hoy en día se necesitan completar. Entre las más necesarias se encuentran: La creación de distintos tipos de usuarios, tal que cada uno tenga una paleta de actividades distintas según su cargo.
\subsection{Establecer la Solución}
Se opta por una completa reingeniería del proyecto, en este caso se utilizará Java para generar el código de este sistema. Se mantendrán algunas cosas que vienen ya hechas en el proyecto anterior, pero no se asegura la integridad total de las estructuras anteriores, esto quiere decir que si es necesario se hará un cambio fuerte en la base de datos para poder sobrellevar los nuevos requerimientos, estos requerimientos y los anteriores estarán en armonía.
\subsection{¿Por qué ocupar Java?}
El proyecto es construido a través del compendio de tecnologías que engloba la convención J2EE, ya que de esta manera nos enfocamos en el proceso productivo de una manera bastante rígida, gracias a esta forma de trabajo podemos automatizar varios procesos y trabajar de una manera bastante veloz en un sin fin de cosas que nos facilitan el trabajo y el desarrollo del proyecto. En la actualidad no sólo J2EE, si no también otras tecnologías asociadas a Java, eso facilita la reconstrucción del proyecto de manera más expedita y rápida. Todo con el fin de realizar de la mejor manera posible este proyecto que beneficia a la universidad.
\subsection{¿Por qué no otras tecnologías?}
En comparación con PHP el cual nos permite crear la pagina de manera rápida e incorporando con un framework la unidad de persistencia. Pero como desventaja esta el concepto de no tener elasticidad para hacer otras cosas. Parámetro por el cual Ruby on Rails podría llevar a cabo de mejor manera. Pero el punto frágil de Ruby es que como utiliza un Framework de propósito general (igual que PHP) su capacidad de aprendizaje es bastante costoso.

% enfocar a las cosas que no se pueden haceren otros lenguajes.
\section{Ciclo de vida del Software}
\begin{enumerate}
\item Toma de Requerimientos: Proceso que básicamente comprende 2 actividades, primero la toma de requerimientos en base al sistema SEPA anterior, segundo una serie de entrevistas para captar las historias de usuarios que posteriormente se convierten en requerimientos.
\item Análisis de Requerimientos: Etapa en donde el profesor encargado discute con el alumno cuales requerimientos deben realizarse, de acuerdo con las expectativas y las necesidades del equipo de trabajo.
\item Diseño de datos: Proceso en donde se desarrolla el modelo de datos que debe soportar a los requerimientos, cada parte del modelo corresponde a un grupo de requerimientos aceptados como una ramificación completa del sistema a desarrollar.
\item Diseño Estructural: En este punto el alumno trabaja desarrollando el proyecto en coordinación con el profesor encargado y este va revisando constantemente el avance, emitiendo preferencias y aclaraciones.
\item Pruebas: Proceso por el cual el sistema en una fase anterior al termino de proyecto se encuentra en estudio, el profesor encargado verifica la fiabilidad y calidad de los requerimientos ya resueltos.
\item Instalación: En este punto se produce la incorporación del sistema en su ambiente por defecto, lo que implica un termino del proyecto, solo dentro del marco de trabajo de título para el alumno de este proyecto.
\end{enumerate}
\section{Historia del Arte}
La cantidad de datos que se obtienen en un establecimiento universitario, es de vital importancia para el completo desarrollo de sus actividades, ya que permiten una normal toma de decisiones que afectan positivamente el crecimiento de la institución académica, ya sea de forma docente como administrativa. 

Una de las mas importantes finalidades que contempla un estudio estadístico, responde a la necesidad de los estudiantes, las familias y los profesores por elegir el mejor plan de estudios que el mercado puede ofrecerles, de esta manera, es necesario señalar que todas las entidades educacionales debieran tener la misión de informar acerca de su situación en el grupo de universidades al cual representa.

De igual manera, los estudios estadísticos son importantes para el desarrollo de la entidad educacional, ya que responden a la necesidad propia de la universidad de ver como se encuentra dentro del mercado. En pos de mejorar las condiciones y estatutos actuales. Buscando como objetivo llegar más cerca de su visión como organización.

En chile esta creciendo la necesidad de que las instituciones deben estar evaluadas por organismos preparados para testificar las características de su desempeño y los profesionales que produce. Para estos efectos, existen organismos estatales que acreditan lo que cada universidad propone, pero en otro lado existen entandares de calidad que son medidos por organismos extranjeros, para dar un sello de calidad, evaluado por datos estadísticos\cite{data1}.

En 2009 se abre el sitio QS Top Universities que hasta el día de hoy es un sitio altamente eficiente de evaluación a trabes de ránkings enfocados a entregar información a gente proveniente de post-grados, MBA, estudios de ingeniería y negocios. Desarrollando estadísticas de alto alcance en sitios web, eventos, guias y programas por Internet y finalmente entregando soluciones técnicas a otros organismos, tales como universidades o empresas privadas, entre otras.

Sus datos manejan información de 41 países, entre los que están: Argentina, Ecuador, Japon, Singapur, Australia, Egipto. Su servicio esta libre para todo el publico y permite un tipo de suscripción gratuita para visualizar tablas comparativas de manera bien simple.

Permite hacer comparaciones por categorías varias, ya sea país, estado, ciudad y otras características internas de la universidad como ponderaciones o prestaciones de alumnos, tales como cantidad de alumnos o la segregación por sexo o etnia en cada localidad.

Su objetivo es ser el sitio principal y mundial de ránkings de educación para los profesionales ambiciosos que buscan fomentar tanto su desarrollo personal y profesional. Acentuando sus características en entregar soluciones on-line de desarrollo y visualización con alcances estadísticos y comparativos a quienes estén dentro de su sistema.

Una debilidad de este sistema es que solo hace comparaciones de carácter global y no se involucra en temas más finos sobre cada universidad, solo tiene datos recopilados de otras investigaciones para saldar ese tema.

Es una empresa de tamaño mediano, con más de 150 empleados en oficinas de todo el mundo. También se realizan Tours en 70 ciudades en 39 países que permiten a más de 50.000 espectadores, los cuales buscan entrar a una universidad\cite{data2}.

El Ránking de Calidad de las Universidades Chilenas, es un trabajo realizado por AméricaEconomía Intelligence en el 2010, un sitio de ránking para instituciones educacionales en Chile, que utiliza ciertos criterios, como lo son la cantidad de investigaciones, o la masa de estudiantes que entran en cada universidad.

Algunos datos generales que se obtienen, son que la Universidad de Chile y la PUC, tienen diferencias muy pequeñas en el campo de la investigación, para lo cual se mostró que la Universidad de Chile aporto con 1.329 papers publicados, en cambio la PUC hizo un aporte total de 1.029 publicaciones.

Mucha de esta información estadística es recopilada a trabes de encuestas enviadas a distintas universidades y organismos de educación en chile. De esta manera se puede obtener información importante para ser mostrada al publico que desea tomar una decisión en el mercado.

Este sistema da cuenta de una necesidad por parte de las universidades, de ser mas competitiva en el mercado universitario. Aumentando los esfuerzos por adquirir más estudiantes, mostrando mejores profesores, alumnos e investigación.

La gran debilidad de este sistema es que no se efectúan muchos criterios de comparación y por tanto solo da una arista para presentar la comparación lo que resulta incomodo al no poder ver desde otra perspectiva este sistema.

Finalmente se nota a las universidades actuales buscando organismos extranjeros capaces de certificar las características de la educación que ofrecen, de esta manera muestran no solo una acreditación por un organismo chileno, si no también una necesidad por competir a nivel internacional\cite{data3}.

\section{Justificación del Proyecto}
La necesidad de la Universidad Tecnológica Metropolitana por tener un buen sistema de análisis estadístico, radica básicamente en el proceso de acreditación de la misma, lo que hace necesario que la universidad tenga una herramienta de mantenimiento de la información y que estos datos permitan hacer la toma de decisiones un proceso justificado sobre la base de la necesidad de la Universidad.

También es relevante señalar que no cualquier sistema puede ser el mas apropiado para incorporar a la universidad, pues el sistema que se esta realizando actualmente tiene como propósito especifico un cliente y ese usuario particular es la Universidad, de tal manera que en este proyecto no habrá mas indicadores de los que se requieren, asegurando una utilidad en su totalidad, puesto que nada se creará si no existe un requerimiento de la universidad de por medio.

La otra justificación, no es competencia del trabajo de Título, pero es necesario explicarla de manera adecuada. La razón económica, prima por otras ya que un sistema de esta envergadura requiere no solo dinero si no también el consentimiento de varias entidades de poder dentro de la universidad, para realmente tomar la decisión de comprar un sistema externo.

Finalmente el punto negativo de todo esto, es que este trabajo esta acotado dentro de los margenes que debe tener un trabajo de Título, por lo tanto no es posible seguir en el desarrollo a menos que halla una visión de futuro ya sea por parte del profesor a cargo de este proyecto y el alumno que realiza este trabajo.
\section{Análisis de las herramientas}
El desarrollo de las aplicaciones debe ser con un lenguaje que nos permita la correcta ejecución de las tareas de diseño y desarrollo. Por lo cual resulta interesante desarrollar algunos criterios que nos mostrarán las características y la justificación de nuestra elección a la hora de pensar en un lenguaje apropiado para este proyecto.

\begin{enumerate}

\item Sencillez: Característica del lenguaje que hace que sea fácil el crear o entender el código, al momento de programarlo o leerlo. De tal manera que ciertos lenguajes son más explícitos en la forma de mostrar sus sentencias y estructuras de control\cite{data4}.

\item Robustez: Capacidad interna del lenguaje de proporcionar herramientas que permiten minimizar los errores producidos por el programador\cite{data4}.

\item Seguridad: Característica que hace que un lenguaje no permita tocar accesos a donde la aplicación no debiese ir (en la mayoría de los casos), ejemplos como, el acceso ilegal a memoria, ejecución de pruebas o derechos de accesos\cite{data4}.

\item Portabilidad: Capacidad de un lenguaje para correr en múltiples sistemas operativos\cite{data4}.

\item Neutralidad: Independencia de la maquina en donde se ejecuta, haciendo una experiencia lo mas similar posible entre distintas máquinas y sistemas operativos\cite{data4}.

\item Threads: manejo de ejecuciones en hilos o programación multi-tarea\cite{data4}.

\item Garbage: Posee un sistema propio de recolección de basura, delegando al propio lenguaje la tarea del manejo de memoria\cite{data4}.

\item Interfaz: Capacidad de producir fácilmente interfaces cómodas para el usuario final\cite{data4}.

\item Expresividad: Cantidad de lineas con las que una acción puede ejecutarse dentro del código\cite{data5}.

\item Compilación: Se refiere al tiempo que toma poder compilar el código producido.

\item Aprendizaje: Dificultad para aprender el lenguaje. Basado simplemente en la experiencia propia sobre cada lenguaje.

\item Estructuras de control: Un lenguaje debe proveer estructuras simples de control, pero tampoco llenarse de estructuras que nunca se van a utilizar\cite{data6}.

\item Abstracción: Capacidad de convertir cosas difíciles en algo simple con la estructura del lenguaje\cite{data6}.

\end{enumerate}

También es importante tener en cuenta el lenguaje a utilizar en el momento de la creación del proyecto, de esta manera se pondrán los lenguajes candidatos para conocerlos en mayor detalle.

\begin{enumerate}

\item PHP: Es el lenguaje de lado servidor más extendido en la web. Nacido en 1994, se trata de un lenguaje de creación relativamente reciente, aunque con la rapidez con la que evoluciona Internet parezca que ha existido toda la vida. Es un lenguaje que ha tenido una gran aceptación en la comunidad de desarrolladores, debido a la potencia y simplicidad que lo caracterizan, así como al soporte generalizado en la mayoría de los servidores\cite{data7}.

\item .NET: Es un framework de Microsoft que hace un énfasis en la transparencia de redes, con independencia de plataforma de hardware y que permita un rápido desarrollo de aplicaciones. Basado en ella, la empresa intenta desarrollar una estrategia horizontal que integre todos sus productos, desde el sistema operativo hasta las herramientas de mercado\cite{data8}.

\item Ruby on Rails: Es un entorno de desarrollo web de código abierto que está optimizado para satisfacción de los programadores y de la productividad. Te permite escribir un buen código favoreciendo la convención antes que la configuración\cite{data9}.

\item Java: Es un lenguaje de programación con el que podemos realizar cualquier tipo de programa. En la actualidad es un lenguaje muy extendido y cada vez cobra más importancia tanto en el ámbito de Internet como en la informática en general. Está desarrollado por la compañía Sun Microsystems con gran dedicación y siempre enfocado a cubrir las necesidades tecnológicas más punteras\cite{data10}. 
\end{enumerate}

En la siguiente tabla muestra una comparación de las cuatro tecnologías anteriormente presentadas de forma explicita para comprender nuestra elección. Es claro decir que uno de los aspectos que marca profundamente la elección sobre el lenguaje con que se va a realizar este proyecto, es determinada por los conocimientos que tiene el alumno y las indicaciones del profesor, en pos de realizar lo más rápido posible el trabajo, haciéndolo de forma elegante. 

\begin{tabular}{| l | l | l | l | l |}
\hline
 & \includegraphics[scale=0.08]{images/icons/php.png} & \includegraphics[scale=0.08]{images/icons/net.png} & \includegraphics[scale=0.51]{images/icons/ruby-on-rails.png} & \includegraphics[scale=0.1]{images/icons/java.png}\\
\hline
\textit{Sencillez} & \textcolor{green}{\CheckmarkBold} &  \textcolor{red}{\XSolidBold} &  \textcolor{red}{\XSolidBold} & \textcolor{green}{\CheckmarkBold}\\
\hline
\textit{Robustez} & \textcolor{red}{\XSolidBold} & \textcolor{green}{\CheckmarkBold} & \textcolor{green}{\CheckmarkBold} & \textcolor{green}{\CheckmarkBold}\\
\hline
\textit{Seguridad} & \textcolor{red}{\XSolidBold} & \textcolor{red}{\XSolidBold} & \textcolor{green}{\CheckmarkBold} & \textcolor{green}{\CheckmarkBold}\\
\hline
\textit{Portabilidad} & \textcolor{green}{\CheckmarkBold} & \textcolor{red}{\XSolidBold} & \textcolor{red}{\XSolidBold} & \textcolor{green}{\CheckmarkBold}\\
\hline
\textit{Neutralidad} & \textcolor{green}{\CheckmarkBold} & \textcolor{red}{\XSolidBold} & \textcolor{green}{\CheckmarkBold} & \textcolor{green}{\CheckmarkBold}\\
\hline
\textit{Threads} & \textcolor{green}{\CheckmarkBold} & \textcolor{red}{\XSolidBold} & \textcolor{red}{\XSolidBold} & \textcolor{green}{\CheckmarkBold}\\
\hline
\textit{Garbage} & \textcolor{red}{\XSolidBold} & \textcolor{green}{\CheckmarkBold} & \textcolor{red}{\XSolidBold} & \textcolor{green}{\CheckmarkBold}\\
\hline
\textit{Interfaz} & \textcolor{red}{\XSolidBold} & \textcolor{green}{\CheckmarkBold} & \textcolor{green}{\CheckmarkBold} & \textcolor{green}{\CheckmarkBold}\\
\hline
\textit{Expresividad} & \textcolor{green}{\CheckmarkBold} & \textcolor{red}{\XSolidBold} & \textcolor{green}{\CheckmarkBold} & \textcolor{red}{\XSolidBold}\\
\hline
\textit{Compilación} & \textcolor{green}{\CheckmarkBold} & \textcolor{red}{\XSolidBold} & \textcolor{green}{\CheckmarkBold} & \textcolor{red}{\XSolidBold}\\
\hline
\textit{Aprendizaje} & \textcolor{green}{\CheckmarkBold} & \textcolor{red}{\XSolidBold} & \textcolor{green}{\CheckmarkBold} & \textcolor{green}{\CheckmarkBold}\\
\hline
\textit{E. de control} & \textcolor{green}{\CheckmarkBold} & \textcolor{green}{\CheckmarkBold} & \textcolor{green}{\CheckmarkBold} & \textcolor{green}{\CheckmarkBold}\\
\hline
\textit{Abstracción} & \textcolor{green}{\CheckmarkBold} & \textcolor{red}{\XSolidBold} & \textcolor{red}{\XSolidBold} & \textcolor{green}{\CheckmarkBold}\\
\hline
\textbf{Total} & 9 & 4 & 8 & 11\\
\hline
\end{tabular}

Como lo muestra la tabla anterior, java es el que puede cumplir más criterios, en el sentido de la observación y el criterio subjetivo del alumno que realiza el trabajo.

\section{Criterios de Calidad}

Se mostrarán a continuación los criterios de calidad que el proyecto como tal va a proporcionar a la universidad. Estos criterios muestran de forma justificada, que la mejor elección es la que nosotros proponemos para la organización, pensando en los costes de tiempo y recursos económicos que un proyecto de esta clase puede significar.

\begin{enumerate}

\item Mantenibilidad: Característica que hace que nuestro proyecto sea fácil de corregir en caso de fallos, mas que nada se refiere a si las bases de nuestro proyecto están debidamente echas y revisadas. Basándose en la concepción de que un proyecto que tiene un modelo bien establecido es más fácil de reparar\cite{data11}.

\item Flexibilidad: Característica que permite a nuestro proyecto un fácil modelo de cambios o adiciones de más características, incluyendo también el echo de que como es un sistema echo dentro del contexto de la Universidad esta lo suficientemente enfocado en ella, pero también es posible cambiar algunas cosas en pos de que este sistema funcione también en otras universidades.\cite{data11}.

\item Testeabilidad: Esto nos muestra que el modelo del sistema ha pasado por test antes de ser ocupado en el proyecto, de manera tal como es.\cite{data11}.

\item Portabilidad: Ya que como se trabaja con Java, el proyecto es básicamente portable a casi cualquier tipo de sistema operativo, lo que lo hace un ente fácil de cambiar en otros entornos de trabajo y es perfectamente acomodable en ese sentido.\cite{data11}.

\item Integridad: Este sistema se caracteriza por su integridad con los datos, lo que nos muestra un dedicado trabajo sobre el tema de la base de datos.\cite{data11}.

\item Rapidez: El sistema esta echo de tal manera que debe funcionar de forma expedita y veloz, sobretodo con la enorme cantidad de datos que se van a manejar.\cite{data11}.

\end{enumerate}


\bibliographystyle{plain}
\bibliography{bibliografia.bib}
\end{document}